\documentclass[12pt, a4paper]{article}
\usepackage[tmargin=1.0cm, bmargin=2.5cm, lmargin=2cm, rmargin=2cm]{geometry}
\usepackage[utf8]{inputenc}\DeclareUnicodeCharacter{2212}{-}
\usepackage[T1]{fontenc}
\usepackage{float}
\usepackage{lmodern}
\usepackage{hyperref}
\hypersetup{
colorlinks = true,
linkcolor  = black,
citecolor = black
}
% Bib stuff
\usepackage[
    backend=biber,
    style=apa,
]{biblatex}
\setlength{\bibhang}{0pt}
\setlength{\bibitemsep}{6pt}
\addbibresource[]{../peak_water.bib}
\usepackage{amsmath}
\usepackage{amssymb}
\usepackage{gensymb}
\usepackage{upgreek}
\usepackage{enumitem}
\usepackage{graphicx}
\usepackage{subcaption}
\graphicspath{{../plots/}}
\usepackage{xcolor, colortbl}
% Setup for the captions
\usepackage[hypcap=true,font={it}]{caption}
\captionsetup{belowskip=2pt, aboveskip=2pt}
\author{Erik Holmgren \\ Advisor: Fabien Maussion}
\title{Master thesis proposal: Peak water using the Open Global Glacier Model}
\date{March 2021}
\begin{document}
\maketitle
\noindent
Mass loss from glaciers has increased during the second half of the 20th century
\parencite{vaughanObservationsCryosphere2013} and is prediced, in all current
climate projections, to continue throughout the 21st century
\parencite{ipccClimateChange20142014}. The magnitude of the end of century
glacial mass loss varies greatly depending on the region and climate scenario --
\textcite{hussNewModelGlobal2015} found a global glacier volume decrease between
25\% (RCP2.6) and 48\% (RCP8.5) and regional losses between 20 and 90\%.

% Simulating global glaciers \textcite{hussNewModelGlobal2015} found that the
% global glacier volume will decrease with 25$\pm$5\% for RCP2.6, 33$\pm$8\% for
% RCP4.5 and 48$\pm$9\% by the end of the century. Furthermore, they found large
% regional differences -- glaciers in central Europe and in low latitudes may lose
% as much as 90\% of their mass while glaciers in Arctic Canada and
% Antarctica/Subantarcitca lose 20\%. 

% End of century glacier mass exhibit a substantial spread depending on the GCM
% used for the emission scenario in the simulation. For instance, mass losses in
% Svalbard varies between 12\% and 90\% for RCP4.5
% \parencite{hussNewModelGlobal2015}.

Glaciers play an important role as storage magasins, delaying up to 79\% of the
total precipitation falling on the glacier surface (Aral Basin) as meltwater
runoff later in the season. The benefits of this seasonal delay is particularly
important in regions with a warm and dry ablation season
\parencite{kaserContributionPotentialGlaciers2010}. One of these areas is the
Indus basin where, during the pre-monsoon season, up to 60\% of the total
irrigation volume comes from either snow or glacier melt -- resulting in an 11\%
increase of the total crop production
\parencite{biemansImportanceSnowGlacier2019}. The Indus basin is one example of
large river basins which under the present climate experiences water scarcity --
threatening the food security for millions of people
\parencite{kummuClimatedrivenInterannualVariability2014}. The populated areas on
the dry, western, slopes of the Andes are other examples of regions depending on
glacier meltwater for potable water and power generation.
\textcite{vergaraEconomicImpactsRapid2007} estimate the cost of mitigation and
adaption to retreating glaciers in the Andes between US\$300 million up to US\$
1.5 billion.
% of glacier melt include significant contributions to sea level rise (e.g.
% \cite{marzeionFutureSealevelChange2012a})

% More direct societal impacts The Indus basin is experiencing water scarcity under the present climate
% \parencite{kummuClimatedrivenInterannualVariability2014}.

% Out of the total precipitation falling onto the glacier surface, between 79 and
% 17\% will experience a seasonal delay -- meaning that the water will be stored
% in the glacier and released later in the season. The relative importance of
% glacier melt water to the basin water availability decreases in the presence of
% liquid precipitation, hence the positive effects of seasonally delayed water
% release from glaciers are more pronounced in areas experiencing warm and dry
% ablation seasons \parencite{kaserContributionPotentialGlaciers2010}.

% The monthly percentage of total water input to a basin that experience seasonal
% delayed water release by glaciers decreases downstream the river while the
% population generally increases. Thus, the societal impacts of seasonal delayed
% water release by glacier melt reaches a maximum at intermediate altitude bands
% \parencite{kaserContributionPotentialGlaciers2010}.




\printbibliography
\end{document}